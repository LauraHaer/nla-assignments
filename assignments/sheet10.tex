%%%%%%%%%%%%%%%%%%%%%%%%%%%%%%%%%%%%%%
% Numerical Linear Algebra class 2022 
% Sheet 10
%%%%%%%%%%%%%%%%%%%%%%%%%%%%%%%%%%%%%%

\begin{Sheet}[to be handed in until January 20, 2023, 2pm.]
  \label{sheet10}

  \begin{Problem}[{\cite[Exercise 6.21]{Saad00}}]
    % Saad Aufgabe 21 (S. 214)
    Consider a matrix of the form:
    \begin{gather*}
      \mata = \id + \alpha \matb
    \end{gather*}
    where $\matb$ is skew-symmetric (real), that is,
    $\matb^T = -\matb$.
    \begin{enumerate}[(a)]
    \item Show that
      \begin{gather*}
        \frac{\scal(\mata\vx,\vx)}{\scal(\vx,\vx)} = 1
      \end{gather*}
      for all nonzero $\vx$.
    \item\label{sheet10:problem1:partb} Consider the Arnoldi process
      for $\mata$. Show that the resulting Hessenberg matrix will have
      the following tridiagonal form
      \begin{gather*}
        \matH_m =
        \begin{pmatrix}
          1&-\eta_2&&& \\
          \eta_2&1&-\eta_3&& \\
           &\ddots&\ddots&\ddots& \\
           &&\eta_{m-1}&1&-\eta_m \\
           &&&\eta_m&1
        \end{pmatrix}
        .
      \end{gather*}
    \item Using the result of part \eqref{sheet10:problem1:partb},
      explain why the conjugate gradient method applied to a linear
      system with the matrix $\mata$ will still yield residual vectors
      that are orthogonal to each other.
    \item Can this algorithm break down before the solution is
      reached?
    \end{enumerate}
  \end{Problem}

  \begin{Problem}[{\cite[Exercise 6.8]{Saad00}}]
    Show how GMRES (Algorithm 2.3.41 in the lecture notes) and Arnoldi
    with modified Gram-Schmidt (Algorithm 2.3.29 in the lecture notes)
    will converge on the linear system $\mata\vx=\vb$ when
    \begin{gather*}
      \mata =
      \begin{pmatrix}
        &&&&1\\
        1&&&&\\
        &1&&&\\
        &&1&&\\
        &&&1&
      \end{pmatrix},
      \quad
      \vb =
      \begin{pmatrix}
        1\\ 0\\ 0\\ 0\\ 0
      \end{pmatrix},
    \end{gather*}
    and $\vx_0 = \boldsymbol 0$.
  \end{Problem}

  \begin{Problem}[Programming]
    \hfill\\\vspace{-4ex}
    \begin{enumerate}[(a)]
    \item Implement the conjugate gradient method (Algorithm 2.3.57 in
      the lecture notes).
    \item Use your implementation to solve the 2D Laplace problem
      \begin{gather*}
        \matl_2\vx=\vb
      \end{gather*}
      as in \cref{sheet7:problem4} on \cref{sheet7} with right hand
      side vector $\vb=(1,...,1)^T$ and initial guess
      $\vx^{(0)}=(0,...,0)^T$ with $n=50$ and $n=100$. Observe the
      convergence for the first 50 steps. What convergence rate do you
      expect? Compare your results of the conjugate gradient method
      with the steepest decent method from \cref{sheet8:problem4} of
      \cref{sheet8}.
    \end{enumerate}
  \end{Problem}

  \vfill
  \bibliographystyle{alpha}
  \bibliography{bib}
\end{Sheet}


%%% Local Variables: 
%%% mode: latex
%%% TeX-master: "main"
%%% End: 
